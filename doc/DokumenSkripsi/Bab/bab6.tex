\chapter{Kesimpulan dan Saran}
\section{Kesimpulan}
\begin{enumerate}
  \item OSMXML merupakan data peta yang disediakan oleh situs
  OpenStreetMap dalam bentuk dokumen XML. OSMXML memiliki informasi node dan
  edge, informasi tersebut dapat dibaca menggunakan Javascript dan dimodelkan 
  ke dalam bentuk graf berarah, graf tersebut dimodelkan menggunakan
  \textit{adjacency list}.
  
  \item Algoritma Dijkstra dapat mencari rute terdekat pada graf berarah.
  Aplikasi menggunakan algoritma dijkstra pada graf berarah yang sudah
  dimodelkan.
  
  \item Visualisasi graf dapat dibuat menggunakan Google Maps Javascript API.
  Aplikasi menampilkan informasi node dan edge pada OSMXML dengan objek
  \textit{marker} dan \textit{polyline}. Rute terdekat didapatkan dari hasil
  algoritma dijkstra yang telah diimplementasikan dan divisualkan dengan
  \textit{polyline}.
\end{enumerate}

\section{Saran}
OSMXML memiliki informasi selain node dan edge yaitu ``relation''.
\textit{Relation} menyimpan informasi seperti rute angkutan, rute bus, rute
\textit{hiking}, dan lain-lain. Untuk pengembangan aplikasi yang juga
menggunakan data peta dari OpenStreetMap dapat menggunakan informasi tersebut,
sehingga tidak hanya rute mengemudi saja, tetapi juga dapat mencari rute
terdekat angkutan, bus, atau rute lainnya.
